\documentclass{article}

%\usepackage{natbib}
\usepackage[square,sort,comma,numbers]{natbib}
\usepackage{parskip}

\begin{document}

\title{A Comparison of Parallel Parallel Integer Sorting Algoritms}
\author{Patrick Collins}
\date{December 3, 2013}
\maketitle

% \begin{abstract}
% Abstract goes here...
% \end{abstract}

\section{Introduction}

\section{Implementation}
Talk about stuff bar

\subsection{Program Structure}
Blah blah

\subsubsection{Storage Duration}

Before implementing any sorting algorithm, it is necessary to
determine the most effective way to store the data to be
sorted. The C11 standard defines four diferent storage durations:
static, automatic, thread and allocated (\cite[\S 6.2.4]{CStd}).
Thread variables are new in C11 and irrelevant for our purposes, since they cannot be
accessed from outside of the thread in which they are declared. That
leaves three candidate storage durations: static, automatic and
allocated. Memory with static duration is allocated either when a
variable is declared outside of a
function, or when a variable is declared inside of a function with the \texttt{static}
keyword. Automatic memory is allocated when a varaible is declared
inside of a function without the \texttt{static} keyword, and memory
with the `allocated` storage duration is allocated with the
\texttt{malloc()} function or similar. \smallskip

%types of memory? i don't know what language to use here
The C11 standard is silent, however, on how these different types of
memory are implemented. While allocated memory is most naturally
associated with the heap and automatic memory with the stack, the C11
standard does not contain the words ``stack'' or ``heap,'' leaving the
decision of how to implement each type to the compiler. In order to
determine the relative efficiencies of each type, then, it is
necessary to compare each on a realistic workload.

As a baseline, the single-threaded bubble sort was used to compare the
use of an integer array declared locally in the \texttt{main()}
function, an integer array declared in the global scope, and an
integer array allocated with the \texttt{malloc()} function. The
source code for each configuration is available in \texttt{bsort.c},
\texttt{bsort-global.c} and \texttt{bsort-malloc.c}. \bigskip

The results are as follows:
\begin{center}
\begin{tabular}{lllll}
 Implementation  &  Size & Average  &  Median     &  Standard Deviation             \\
 bsort           &  small    &  0.2238 s   &  0.2182 s            &  0.0093 s  \\
 bsort           &  medium   &  0.2220 s   &  0.2180 s            &  0.0085 s  \\
 bsort           &  large    &  23.4886 s  &  23.5006 s           &  0.6830 s  \\
 bsort global    &  small    &  0.2475 s   &  0.3389 s            &
 0.1960 s  \\
bsort global    &  large    &  30.6455 s  &  30.6182 s           &  0.4994 s  \\
 bsort global    &  medium   &  0.3068 s   &  0.3111 s            &  0.0131 s  \\
 bsort malloc    &  small    &  0.3429 s   &  0.4268 s            &  0.1703 s  \\
 bsort malloc    &  medium   &  0.3300 s   &  0.3306 s            &
 0.0132 s  \\
bsort malloc    &  large    &  32.0951 s  &  32.1063 s           &  0.4550 s  \\
\end{tabular}
\end{center}\bigskip

This indicates that using an array stored in a local variable with automatic storage
duration provides a significant performance advantage over using
either a global variable or dynamically allocated memory.

% why do we care that they're not implemented with a stack?
\subsubsection{Function Call Overhead}
With some of the algorithms discussed later -- in particular,
quicksort and mergesort -- there is no way to avoid the overhead of
many function calls. In an iterative algorithm like bubble sort,
however, it is necessary to choose between inserting the code directly inside
the \texttt{main()} function or splitting it off into a separate
function. The C11 standard again does not specify that functions be
implemented with a stack (\cite[\S 6.7.6.3]{CStd}), although it does
include the \texttt{inline} keyword (\cite[\S 6.4.1, \S
6.7.4]{CStd}). The GCC Manual asserts that \texttt{inline} functions
are just as fast as writing a macro to manually inline the code in
question (\cite[\S 6.39]{GCCMan}).

A comparison of the three options is below:

\begin{center}
\begin{tabular}{lllll}
 Implementation  &  Average  & Size &  Median     &  Standard Deviation  \\
 bsort           &  small    &  0.2238 s   &  0.2182 s            &  0.0093 s  \\
 bsort           &  medium   &  0.2220 s   &  0.2180 s            &  0.0085 s  \\
 bsort           &  large    &  23.4886 s  &  23.5006 s           &  0.6830 s  \\
 bsort factored  &  small    &  0.1757 s   &  0.1966 s            &  0.1737 s  \\
 bsort factored  &  medium   &  0.2576 s   &  0.2586 s            &
 0.0056 s  \\
bsort factored  &  large    &  26.2215 s  &  26.1377 s           &  0.3313 s  \\
 bsort inline    &  small    &  0.1976 s   &  0.3117 s            &  0.1594 s  \\
 bsort inline    &  medium   &  0.2660 s   &  0.2668 s            &
 0.0026 s  \\
bsort inline    &  large    &  26.5490 s  &  26.5511 s           &  0.0266 s  \\
\end{tabular}
\end{center}

These figures do indicate a performance decrease after refactoring
the code to use separate. However, this is likely not a
measurement of the overhead of function calls, but rather the overhead
of dereferencing a pointer to the array each time. Changing to
reference implementation to use a pointer to modify the array inside
of the main function results in the following:

\begin{center}
\begin{tabular}{lllll}
 Implementation  &  Size    &  Average    &  Median     &  Standard Deviation  \\
 bsort pointer   &  small   &  0.1731 s   &  0.2046 s   &  0.1656 s            \\
 bsort pointer   &  medium  &  0.2685 s   &  0.2693 s   &  0.0031 s            \\
 bsort pointer   &  large   &  26.8550 s  &  26.8656 s  &  0.0692 s            \\
\end{tabular}
\end{center}

These measurements are identical to the measurements above from the
factored and inline versions, confirming that the slowdown is due to
the additional pointer.

One possible solution to this is to copy the array into a local
variable at the start of the algorithm, perform the sorting on the
local variable, and then copy the local variable back into the
original array. This implementation resulted in the following speeds:

\begin{center}
\begin{tabular}{lllll}
 Implementation     &  Size    &  Average    &  Median     &  Standard Deviation  \\
 bsort pointer opt  &  medium  &  0.2723 s   &  0.2722 s   &  0.0061 s            \\
 bsort pointer opt  &  small   &  0.1892 s   &  0.2287 s   &  0.1796 s            \\
 bsort pointer opt  &  large   &  27.0129 s  &  26.9419 s  &  0.2426 s            \\
\end{tabular}
\end{center}

\subsubsection{Summary}
These measurements indicate that the optimal structuring of the
program is to use local, automatic variables to store the array to be
sorted. They also indicate that refactoring code to remove the body of
the algorithm from the \texttt{main()} function is associated with
some overhead. However, this overhead is due to dereferencing the
pointer to the array to be sorted, rather than the function call
itself. This is not reduced by copying the array inside of the algorithm.

Results:

PUT A GRAPH HERE LATER


\section{Algorithm}
\subsection{Bubble Sort}
... TODO ...
\begin{itemize}
\item Run time: ...
\end{itemize}

\subsection{Mergesort}
Implementation:

A parallel mergesort was implemented according to the following
algorithm:

  

\bibliography{writeup}
\bibliographystyle{acm}
\nocite{*}


\end{document}